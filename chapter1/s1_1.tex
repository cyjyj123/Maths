\section{命题}
\subsection{命题的概念和定义}
我们在生活中,经常会遇到许多句子,其中有陈述句、感叹句、疑问句等。
在这些句子中,有一类句子可以判断其是否符合事实,这类句子一定是陈述句。
如果它符合事实那么就称它为真,如果不符合事实就称它为假。
我们把这类可以判断真假的句子叫做\textbf{命题}.如果它为真,则称为\textbf{真命题},如果它为假,则称为\textbf{假命题}.命题的真假称为该命题的\textbf{真值}.

\begin{definition}[命题]
    能够判断真假的陈述句。一般用小写字母$p,q...$表示。
\end{definition}

虽然命题一定是陈述句,但是不能判断真假的陈述句不是命题,所以陈述句不一定是命题。

\begin{example}
    假设昨天下雨了,判断以下句子是否是命题,如果是,则是真命题还是假命题?

    (1)昨天下雨了吗?

    (2)昨天下雨了。
    
    (3)昨天没下雨。
\end{example}
\begin{solution}
    (1)是疑问句,所以不是命题;
    
    (2)是真值为真的陈述句,所以是真命题;

    (3)是真值为假的陈述句,所以是假命题。
\end{solution}

等式、不等式这类式子可以视为陈述句,所以只要它能够判断真假(也就是该式子是否成立)就可以算作命题。
如果该式子成立,那么就是真命题,否则就是假命题。

\begin{example}
    判断下列命题的真值:

    (1)直角三角形斜边长度的平方等于其两直角边长度的平方之和。
    
    (2)$1+1>2$

    (3)$1+1=2$
\end{example}
\begin{solution}
    (1)就是毕达哥拉斯定理(勾股定理),是真命题,它的真值为真;
    
    (2)该不等式不成立,所以是假命题,真值为假;
    
    (3)该等式成立,所以是真命题,其真值为真。
\end{solution}

    判断命题为假的方式只要举出一个反例即可,而判断命题为真,要么它是一个公理,要么通过证明。证明的方式多种多样,我们会在之后进行叙述。

    真命题的真值一般记为T、True、1,假命题的真值一般记为F、False、0.

\subsection{逻辑联结词}
一些命题是简单的,我们把它们称为原子命题,一些命题是复杂的,但可以把它们拆分为若干个原子命题,并用一些符号连接起来,这些符号称为\textbf{逻辑联结词},我们把这些复杂的命题称为\textbf{复合命题}。
简而言之,复合命题是由若干个原子命题通过逻辑联结词连接起来构成的。

常用的逻辑联结词有\textbf{蕴含}、\textbf{合取}、\textbf{析取}、\textbf{否定}和\textbf{双条件}。

\begin{definition}[蕴含]
    $p$蕴含$q$,记为$p \to q$,也可读作如果$p$、那么$q$,当且仅当$p$为真、$q$为假时才为假.
\end{definition}

当命题$p$和$q$取不同值时,该复合命题的真值如下表所示:

\begin{tabular}{c|c|c}
    $p$ & $q$ & $p \to q$ \\
    \hline
    0 & 0 & 1 \\
    0 & 1 & 1 \\
    1 & 0 & 0 \\
    1 & 1 & 1
\end{tabular}

\begin{definition}[合取]
    $p$合取$q$记为$p \wedge q$,当且仅当命题$p$和$q$均为真时才为真.
\end{definition}
\begin{tabular}{c|c|c}
    $p$ & $q$ & $p \wedge q$ \\
    \hline
    0 & 0 & 0 \\
    0 & 1 & 0 \\
    1 & 0 & 0 \\
    1 & 1 & 1
\end{tabular}

\begin{definition}[析取]
    $p$析取$q$记为$p \vee q$,当且仅当命题$p$和$q$均为假时才为假.
\end{definition}
\begin{tabular}{c|c|c}
    $p$ & $q$ & $p \vee q$ \\
    \hline
    0 & 0 & 0 \\
    0 & 1 & 1 \\
    1 & 0 & 1 \\
    1 & 1 & 1
\end{tabular}

\begin{definition}[否定]
    和命题$p$相反的命题,记为$\neg p$.
\end{definition}
\begin{tabular}{c|c|c}
    $p$ & $\neg p$  \\
    \hline
    0 & 1  \\
    1 & 0 
\end{tabular}

\begin{definition}[双条件]
    命题$p$和$q$的真值相同,记为$p \leftrightarrow q$.
\end{definition}
\begin{tabular}{c|c|c}
    $p$ & $q$ & $p \leftrightarrow q$ \\
    \hline
    0 & 0 & 1 \\
    0 & 1 & 0 \\
    1 & 0 & 0 \\
    1 & 1 & 1
\end{tabular}

像上面这样,写出各命题所有取值情况时的复合命题的真值的表格,称为\textbf{真值表}.

\begin{example}
    设有命题$p$:太阳在自转,$q$:地球在自转,
    那么:

    (1)$p \to q$所表达的命题为:如果太阳在自转,那么地球在自转。\par
    由于这两个命题都是真命题,所以该复合命题是真命题,虽然太阳是否自转和地球是否自转没有关系。

    (2)$p \wedge q$所表达的命题为:太阳在自转\textbf{且}地球在自转。\par
    这是一个真命题。

    (3)$p \vee q$所表达的命题为:太阳在自转\textbf{或}地球在自转。\par
    这是一个真命题。

    (4)$\neg q$表示:地球不在自转。这显然是一个假命题。

    (5)$p \wedge \neg q$表示:太阳在自转且地球不在自转。\par
    由于只有两个命题均为真时,合取命题才为真,这里左边的命题为真,右边的命题为假,所以该复合命题是一个假命题。

    (6)$\neg p \vee \neg q$表示:太阳不在自转或地球不在自转。\par
    由于两边都是假命题,根据析取的定义,该复合命题是假命题.
\end{example}

一般我们一个结果只能为真或假的变量称为命题变元,例如$p$、$q$等,当确定其真假后(例如上例中为命题$p$赋值为了“太阳在自转”,相当于为命题变元$p$赋值为真),它就是一个命题。通过命题变元、联结词和括号,我们可以按照下述规则构成一个复杂的式子,称之为命题公式:
\begin{definition}[命题公式]
    又称为\textbf{合式公式},一般用大写字母表示

    (1)命题变元是命题公式;

    (2)如果命题$A$是命题公式,那么命题$\neg A$是命题公式;

    (3)如果命题$A$和命题$B$是命题公式,那么命题$(A \wedge B)$、$(A \vee B)$、$(A \to B)$、$(A \leftrightarrow B)$都是命题公式;
    
    (4)有限次地应用上述步骤得到的也是命题公式。
\end{definition}

通俗地讲,一个命题公式中有若干个命题变元,这些变元在没有赋值的情况下,由于不是一个命题,所以此时该命题公式不是一个命题。而当每个命题变元都有了值后,就变成了命题,也就相当于把这些值代入到了命题公式中,于是形成了一个命题。
当该命题公式中的各个命题变元取一个值时,称为该命题公式的一个\textbf{指派};命题公式的一个指派和此时该公式的真值称为该命题公式的一个\textbf{解释}。
由于每个命题变元只可能取2个值,所以对于具有$n$个命题变元的命题公式有$2^n$种取法,因此有$2^n$种指派,$2^n$种解释。

根据上述的定义,我们可以推导出一些运算规律,有了这些规律,我们就可以在已经知道一些命题的真值的情况下,求得一些命题的真值,从而证明该命题为真、或者知道它是一个假命题。

\subsection{命题的逆命题、否命题和逆否命题}
有了逻辑联结词,我们很容易可以讲解这三个概念。 
\begin{definition}[逆命题]
    设有命题$p \to q$,其逆命题为$q \to p$.
\end{definition}
\begin{definition}[否命题]
    设有命题$p \to q$,其否命题为$\neg p \to \neg q$.
\end{definition}
\begin{definition}[逆否命题]
    设有命题$p \to q$,其逆否命题为$\neg q \to \neg p$.
\end{definition}
这里,重要的一点是一个命题的逆否命题和该命题的真值相同。
\begin{theorem}
    $p \to q \iff \neg q \to \neg p$,即命题$p \to q \leftrightarrow \neg q \to \neg p$是永真式.
\end{theorem}
\begin{proof}
    列表即证:

    \begin{tabular}{c|c|c|c|c|c}
        $p$ & $q$ & $\neg q$ & $\neg p$ & $p \to q$ & $\neg q \to \neg p$ \\
        \hline
        0 & 0 & 1 & 1 & 1 & 1 \\
        0 & 1 & 0 & 1 & 1 & 1 \\
        1 & 0 & 1 & 0 & 0 & 0 \\
        1 & 1 & 0 & 0 & 1 & 1
    \end{tabular}
\end{proof}

\section*{习题 \thesection}
