\section{命题}
\subsection{命题的概念和定义}
我们在生活中,经常会遇到许多句子,其中有陈述句、感叹句、疑问句等。
在这些句子中,有一类句子可以判断其是否符合事实,这类句子一定是陈述句。
如果它符合事实那么就称它为真,如果不符合事实就称它为假。
我们把这类可以判断真假的句子叫做\textbf{命题}.如果它为真,则称为\textbf{真命题},如果它为假,则称为\textbf{假命题}.命题的真假称为该命题的\textbf{真值}.

\begin{definition}[命题]
    能够判断真假的陈述句。
\end{definition}

虽然命题一定是陈述句,但是不能判断真假的陈述句不是命题,所以陈述句不一定是命题。

真命题的真值一般记为T、True、1,假命题的真值一般记为F、False、0.不过在数学上,我们一般习惯使用数字,即记为1和0.

为了符号化命题,我们引入命题常元和命题变元的概念,
它们一般均用小写字母$p,q...$等表示。

\begin{definition}
    只能取值为真或假的变量,称为\textbf{命题变元}。
    (亦称\textbf{命题变项})
    
    值为真或假的常量称为\textbf{命题常元}。(亦称\textbf{命题常项})
\end{definition}

例如,有一变量$p$只能取值为真或假,那么它就是命题变元。
而如果是这样的:“$p$:太阳在自转”,那么此时$p$就是一个命题常项。

从上述例子可以看出,为命题变元赋值一个命题之后,它就变成了一个命题常元。
因此命题常元代表着一个命题,而命题变元不是命题。

有时候,我们并不关心具体的命题,因此我们直接使用$1$和$0$来分别代表真命题和假命题,
它们也是命题常项。

\begin{example}
    假设昨天下雨了,判断以下句子是否是命题,如果是,则是真命题还是假命题?

    (1)昨天下雨了吗?

    (2)昨天下雨了。
    
    (3)昨天没下雨。
\end{example}
\begin{solution}
    (1)是疑问句,所以不是命题;
    
    (2)是真值为真的陈述句,所以是真命题;

    (3)是真值为假的陈述句,所以是假命题。
\end{solution}

等式、不等式这类式子可以视为陈述句,所以只要它能够判断真假(也就是该式子是否成立)就可以算作命题。
如果该式子成立,那么就是真命题,否则就是假命题。

\begin{example}
    判断下列命题的真值:

    (1)直角三角形斜边长度的平方等于其两直角边长度的平方之和。
    
    (2)$1+1>2$

    (3)$1+1=2$
\end{example}
\begin{solution}
    (1)就是毕达哥拉斯定理(勾股定理),是真命题,它的真值为真;
    
    (2)该不等式不成立,所以是假命题,真值为假;
    
    (3)该等式成立,所以是真命题,其真值为真。
\end{solution}

    判断命题为假的方式只要举出一个反例即可,而判断命题为真,要么它是一个公理,要么通过证明。证明的方式多种多样,我们会在之后进行叙述。

\subsection{逻辑联结词}
一些命题是简单的,我们把它们称为原子命题,一些命题是复杂的,但可以把它们拆分为若干个原子命题,并用一些符号连接起来,这些符号称为\textbf{逻辑联结词},我们把这些复杂的命题称为\textbf{复合命题}。
简而言之,复合命题是由若干个原子命题通过逻辑联结词连接起来构成的。

常用的逻辑联结词有\textbf{蕴含}、\textbf{合取}、\textbf{析取}、\textbf{否定}和\textbf{双条件}。

\begin{definition}[蕴含]
    $p$蕴含$q$,记为$p \to q$,也可读作如果$p$、那么$q$,当且仅当$p$为真、$q$为假时才为假.
\end{definition}

当命题变元$p$和$q$取不同值时,该复合命题的真值如下表所示:

\begin{tabular}{c|c|c}
    $p$ & $q$ & $p \to q$ \\
    \hline
    0 & 0 & 1 \\
    0 & 1 & 1 \\
    1 & 0 & 0 \\
    1 & 1 & 1
\end{tabular}

\begin{definition}[合取]
    $p$合取$q$记为$p \wedge q$,当且仅当命题$p$和$q$均为真时才为真.
\end{definition}
\begin{tabular}{c|c|c}
    $p$ & $q$ & $p \wedge q$ \\
    \hline
    0 & 0 & 0 \\
    0 & 1 & 0 \\
    1 & 0 & 0 \\
    1 & 1 & 1
\end{tabular}

\begin{definition}[析取]
    $p$析取$q$记为$p \vee q$,当且仅当命题$p$和$q$均为假时才为假.
\end{definition}
\begin{tabular}{c|c|c}
    $p$ & $q$ & $p \vee q$ \\
    \hline
    0 & 0 & 0 \\
    0 & 1 & 1 \\
    1 & 0 & 1 \\
    1 & 1 & 1
\end{tabular}

\begin{definition}[否定]
    和命题变元$p$真值相反的命题变元,记为$\neg p$.
\end{definition}
\begin{tabular}{c|c|c}
    $p$ & $\neg p$  \\
    \hline
    0 & 1  \\
    1 & 0 
\end{tabular}

\begin{definition}[双条件]
    命题变元$p$和$q$的真值相同时为真、不同时为假,记为$p \leftrightarrow q$.
\end{definition}
\begin{tabular}{c|c|c}
    $p$ & $q$ & $p \leftrightarrow q$ \\
    \hline
    0 & 0 & 1 \\
    0 & 1 & 0 \\
    1 & 0 & 0 \\
    1 & 1 & 1
\end{tabular}

像上面这样,写出各命题所有取值情况时的复合命题的真值的表格,称为\textbf{真值表}.

\begin{example}
    设有命题$p$:太阳在自转,$q$:地球在自转,
    那么:

    (1)$p \to q$所表达的命题为:如果太阳在自转,那么地球在自转。\par
    由于这两个命题都是真命题,所以该复合命题是真命题,虽然太阳是否自转和地球是否自转没有关系。

    (2)$p \wedge q$所表达的命题为:太阳在自转\textbf{且}地球在自转。\par
    这是一个真命题。

    (3)$p \vee q$所表达的命题为:太阳在自转\textbf{或}地球在自转。\par
    这是一个真命题。

    (4)$\neg q$表示:地球不在自转。这显然是一个假命题。

    (5)$p \wedge \neg q$表示:太阳在自转且地球不在自转。\par
    由于只有两个命题均为真时,合取命题才为真,这里左边的命题为真,右边的命题为假,所以该复合命题是一个假命题。

    (6)$\neg p \vee \neg q$表示:太阳不在自转或地球不在自转。\par
    由于两边都是假命题,根据析取的定义,该复合命题是假命题.
\end{example}

在上例中,$p$和$q$为命题常项(因为为它们赋值了,因此它们都是命题),所以由它们和联结词构成的是一个复合命题。
如果我们不为$p$和$q$赋值,使它们成为命题变元,那么它们和联结词构成的就是一个真值未定的式子,此时就不是一个命题,我们把这样的式子叫做\textbf{命题公式}(简称为公式),当命题变元的真值确定后,该公式的真值随即确定,既然真值确定了,那么它也就变成了一个命题。

联结词除了能用于命题常元和命题变元直接的运算,也能用于合式公式之间的运算,因为只要参加运算的项确定了真值就可以通过联结词进行运算,而合式公式也具有真值,因此也能进行联结词运算。

一般我们用大写字母表示命题公式。我们还可以使用小括号表示该公式中含有的命题变元,但这不是必须的。例如$A$可以表示一个公式,$A(p,q)$则表示该公式含有命题变元$p$和$q$.
命题常元和命题变元本身也是一个命题公式,所以公式中可能只有1个命题变元,也可能像上面一样,有2个命题变元,也可能有3个等等。

命题公式由命题变元、联结词和括号构成,从左到右运算,先算括号内再算括号外。除此之外,联结词的优先级从高到低为否定、合取、析取、蕴含、双条件。

命题公式是递归定义的,它的具体生成规则如下:
\begin{definition}[命题公式]
    又称为\textbf{合式公式},一般用大写字母表示

    (1)命题常元和命题变元是命题公式;

    (2)如果$A$是命题公式,那么$\neg A$是命题公式;

    (3)如果$A$和$B$是命题公式,那么$(A \wedge B)$、$(A \vee B)$、$(A \to B)$、$(A \leftrightarrow B)$都是命题公式;
    
    (4)有限次地应用上述步骤得到的也是命题公式。
\end{definition}

通俗地讲,一个命题公式中有若干个命题变元,这些变元在没有赋值的情况下,由于不是一个命题,所以此时该命题公式不是一个命题。而当每个命题变元都有了值后,就变成了命题,也就相当于把这些值代入到了命题公式中,于是形成了一个复杂命题。
当该命题公式中的各个命题变元取一个值时,称为该命题公式的一个\textbf{真值指派}(简称指派);
命题公式的一个指派和此时该公式的真值称为该命题公式的一个\textbf{解释}。
由于每个命题变元只可能取2个值,所以对于具有$n$个命题变元的命题公式,
有$2^n$种取法,因此有$2^n$种指派,$2^n$种解释。

\begin{example}
    设有命题变元$p,q,r$判断下列是否是命题公式?
    (1)$p$;
    (2)$p \wedge q$;
    (3)$p \neg \vee q \to r$;
    (4)$p \vee \neq q \to r$;
    (5)$(p \vee \neq q) \to r$;
\end{example}
\begin{solution}
    (1)是公式。因为命题变元本身也是命题公式;
    (2)是公式。
    (3)不是公式。因为$\neg$联结词只能在命题变元的前面;
    (4)是公式。顺便说一句,此时,先算$\neq q$,然后算$\neq \to r$,最后再算$p$析取上一步的结果;
    (5)是公式。和(4)的区别在于先算括号内的,然后在计算结果蕴含$r$.
\end{solution}

命题公式的真值情况也可以使用真值表表示。只要在表格中列出命题公式中含有的所有命题变元和所有指派,并计算每种指派情况下公式的真值即可。

如果两个命题公式$A$和$B$在所有的指派下,其对应的真值总是相同的,那么称它们等价,记为$A \Leftrightarrow B$.
于是,我们可以推导出一些规律,例如$\neg \neg A \Leftrightarrow A$,从而简化后面的计算。

\begin{theorem}(双重否定)
    设有命题公式$A$,则$\neg \neg A \Leftrightarrow A$.
\end{theorem}
\begin{proof}
    列表即证:
    \begin{tabular}{c|c|c}
        $A$ & $\neg A$ & $\neg \neg A$ \\
        \hline
        0 & 1 & 0 \\
        1 & 0 & 1 \\
    \end{tabular}

    我们主要看第一列和第三列,无论$A$取$0$还是$1$,$\neg \neg A$都是对应相等的,因此这两者是等价的。
\end{proof}

主要的运算规律:

(1)交换律:

$A \wedge B \Leftrightarrow B \wedge A$

$A \vee B \Leftrightarrow B \vee A$

% 还有一些之后再写

当然命题变元也是命题公式,所以也是遵循上面的运算规律的。

命题公式主要可以分为三类:
\begin{definition}(永真式)
    设命题公式$A(p,q,...)$,无论$p,q...$取什么值,命题公式A总为真,该命题公式称为\textbf{永真式},又称为\textbf{重言式}.
\end{definition}
\begin{definition}(永假式)
    设命题公式$A(p,q,...)$,无论$p,q...$取什么值,命题公式A总为假,该命题公式称为\textbf{永假式},又称为\textbf{矛盾式}.
\end{definition}
\begin{definition}
    设命题公式$A(p,q,...)$,只要$p,q...$有一组指派使得命题公式A为真,该命题公式称为\textbf{可满足式}.
\end{definition}

显然,永真式一定是可满足式。

根据上述定义,如果有命题公式$A$,且$A \Leftrightarrow 1$,
那么就意味着无论$A$无论取何种指派,其真值都为1,即$A$是永真式;
同理,$A \Leftrightarrow 0$表示$A$是永假式。

\subsection{命题的逆命题、否命题和逆否命题}
有了逻辑联结词,我们很容易可以讲解这三个概念。 
\begin{definition}[逆命题]
    设有命题$p \to q$,其逆命题为$q \to p$.
\end{definition}
\begin{definition}[否命题]
    设有命题$p \to q$,其否命题为$\neg p \to \neg q$.
\end{definition}
\begin{definition}[逆否命题]
    设有命题$p \to q$,其逆否命题为$\neg q \to \neg p$.
\end{definition}
这里,重要的一点是一个命题的逆否命题和该命题的真值相同。在证明它之前,我们先引入几个概念:



\begin{theorem}
    $p \to q \iff \neg q \to \neg p$,即命题$p \to q \leftrightarrow \neg q \to \neg p$是永真式(也就是该式$\Leftrightarrow 1$).
\end{theorem}
\begin{proof}
    (1)第一种方法:列表即证:

    \begin{tabular}{c|c|c|c|c|c}
        $p$ & $q$ & $\neg q$ & $\neg p$ & $p \to q$ & $\neg q \to \neg p$ \\
        \hline
        0 & 0 & 1 & 1 & 1 & 1 \\
        0 & 1 & 0 & 1 & 1 & 1 \\
        1 & 0 & 1 & 0 & 0 & 0 \\
        1 & 1 & 0 & 0 & 1 & 1
    \end{tabular}

    (2)第二种方法:
    证明:$p \to q \leftrightarrow \neg q \to \neg p \Leftrightarrow 1$.

    计算等价号$\Leftrightarrow$左侧:
    $p \to q \leftrightarrow \neg q \to \neg p\Leftrightarrow ((p \to q) \to (\neg q \to \neg p)) \vee ((\neg q \to \neg p) \to (p \to q))$,

    该式子根据析取号分为两项,只要任意一项等价于$1$,该式子就等价于$1$,
    我们计算左边这一项:
    \begin{equation*}
        \begin{aligned}
    & (p \to q) \to (\neg q \to \neg p) \\
    & \Leftrightarrow (\neg (p \to q) \vee (\neg q \to \neg p)) \\
    & \Leftrightarrow \neg (\neg p \vee q) \vee (\neg \neg q \vee \neg p) \\
    & \Leftrightarrow p \wedge \neg q \vee q \vee \neg p  \\
    & \Leftrightarrow p \wedge \neg q \vee \neg p \vee q \\
    & \Leftrightarrow  (p \vee \neg p) \wedge (\neg q \vee \neg p) \vee q \\
    & \Leftrightarrow 1 \wedge (\neg q \vee \neg p) \vee q \\
    & \Leftrightarrow \neg q \vee \neg p \vee q \\
    & \Leftrightarrow \neg q \vee q \vee \neg p \\
    & \Leftrightarrow 1 \vee \neg p \\
    & \Leftrightarrow 1
        \end{aligned}
    \end{equation*}

    即证。
\end{proof}

\subsection{必要条件、充分条件、充分必要条件}
\begin{definition}
    如果命题$p$可以推出命题$q$,则记为$p \Rightarrow q $.

    其中,$p$称为$q$的\textbf{充分条件},$q$称为$p$的必要条件。
\end{definition}

\begin{definition}
    有命题$p,q$,如果$p \Rightarrow q$且$q \Rightarrow p$(即$(p \Rightarrow q) \wedge (q \Rightarrow p)$),
    则记为$p \Leftrightarrow q$.

    称$p$是$q$的\textbf{充分必要条件},简称\textbf{充要条件},
    显然,$q$也是$p$的充要条件。
\end{definition}

充要条件的符号和等价符号是一样的,因为它们本质是一回事。

\begin{example}
    判断下列命题$p$是$q$的什么条件?

    (1)
\end{example}

\subsection{量词}
设有命题“$p$:对于任何实数$x$,$2x$是偶数”,
我们可以得到它的否命题“$\neg p$:存在实数$x$,使得$2x$是奇数”。

上面,我们在命题的条件中描述$x$时用到了一些限定词:“任何”、“存在”。
这些词称为\textbf{量词}.
表示“任何的、所有的”的量词称为\textbf{全称量词},记为$\forall$;
表示“至少存在1个”的量词称为 \textbf{存在量词},记为$\exists$。

我们先引入一个概念,“$x \in R$”表示“$x$是一个实数”。具体的含义会在下一节中说明。

量词一般写在字母前面,例如“对于任何实数$x$”可以写成“$\forall x \in R$”,“存在一个实数$x$”可以写成“$\exists x \in R$”.

\section*{习题 \thesection}
