\section{集合在概率论中的应用}
\subsection{随机现象和随机事件}
概率论是研究随机现象的一门数学分支。所谓随机现象指的是结果不确定的事情,例如“抛掷硬币”,在这件事情发生之前,我们只知道它可能出现正面或者反面而不能知道到底是哪一面。

我们重复进行同一随机现象若干次,将每一次都称为该随机现象的一次\textbf{随机试验},简称试验,一般记为$E$。
每个试验都具有以下性质:
(1)可重复性。即该试验可以重复若干次。
(2)可知性。即每次试验的结果的所有可能在试验前都是已知的,并且每次的所有可能结果都是相同的。
(3)不确定性。即每次试验的结果具体是哪种可能在试验前都是未知的。

我们将试验中的每个可能结果称为该随机试验的一个\textbf{样本点},所有样本点构成的集合称为\textbf{样本空间},记为$\Omega$.

例如,如果用$H$表示正面,$T$表示反面,那么试验“抛掷一枚硬币,观察其正反面”的样本空间则可以记为“$\Omega=\{H,T\}$”,又如试验“抛掷骰子,观察点数”的样本空间为“$\{1,2,3,4,5,6\}$”.

\begin{example}
    用集合表示下列试验的样本空间:
    
    (1)$E_1$:仙女座星系人到太阳系八大行星随机旅游选择的星球;

    (2)$E_2$:公园1小时内的游客数量。
\end{example}
\begin{solution}
    (1)$\Omega_1=\{金星,火星,水星,木星,地球,土星,海王星,天王星\}$(因为是集合,所以顺序无所谓)

    (2)$\Omega_2=\{0,1,2,3,...\}$,或者$\Omega_2=\{x|x \geq 0 , x \in Z\}$
\end{solution}

我们把部分可能发生的结果构成的集合称为\textbf{随机事件},简称事件。例如在抛硬币试验中,$A=\{H\}$表示出现正面的事件,$\{T\}$表示出现反面的事件;抛骰子试验中,$\{1\},\{2\},\{3\},...$表示出现各点数的事件。
而集合$\{1,3,5\}$代表结果为奇数这样一个事件,因此只要集合中的一个元素出现了(一个样本点发生了),就称该事件发生。显然,事件是样本空间的一个子集。

我们将由样本空间中的一个元素构成的集合,称为基本事件。例如上述的$\{H\},\{T\},\{1\},\{2\},...$。由于每次试验,样本空间中至少有一个样本点发生,因此$\Omega$本身也是一个事件,它必然发生,所以$\Omega$是必然事件。
必然事件即一定发生的事件。除此之外,我们将永远不可能发生的事件称为不可能事件。空集$\emptyset$为不可能事件。

\begin{example}
    有试验“抛掷2次硬币,观察正反面”,
    
    (1)列出样本空间和所有的基本事件;
    
    (2)用集合表示事件$A$:“至少有一次出现正面”
\end{example}

\begin{solution}
    (1)$\Omega=\{HH,HT,TH,TT\}$,

    基本事件:
    \{HH\},\{HT\},\{TH\},\{TT\}

    (2)$A=\{HH,HT,TH\}$
\end{solution}

\subsection{随机事件的运算}
由于随机事件就是集合,所以随机事件的运算就是集合的运算,集合的运算这里不再赘述,这里的重点在于集合的运算在概率论中所表示的含义。
设有随机事件$A,B,...$

(1)交集$A \cap B$称为它们的积事件,也记为$A \times B$或$AB$(即乘号可以省略)。

积事件发生代表$A$和$B$都发生,因为交集代表两个集合中相同的元素,因此交集发生,即代表发生的样本点两集合都有,也就是两个事件都发生。
以此类推,$ABC$发生代表三个事件都发生、$ABC...$代表若干个事件都发生。

(2)并集$A \cup B$称为它们的和事件,也记为$A+B$。

和事件发生代表两个事件至少有1个发生(当然包括只有$A$发生、只有$B$发生和$A,B$都发生的情况)。
同上,以此类推$A+B+C+...$代表至少有1个事件发生。

(3)补集,在概率论中我们一般以样本空间作为全集,事件$A$的补集称为它的\textbf{对立事件},记为$\overline{A}$。

\subsection{频率、概率和古典概型}