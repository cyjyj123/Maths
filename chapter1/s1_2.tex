\section{集合}
\subsection{集合的概念和表示}
多姿多彩的世界有着多种多样的物质。无论是什么物质,都是由特定的化学元素构成的。
虽然每一种化学元素的具体性质都不同,但是无论是何种化学元素,我们都可以把它们归为一类——即“化学元素”。
我们把可以归为一类的对象组合在一起构成的总体称为\textbf{集合}.其中的一个对象称为该集合的一个\textbf{元素}.
例如,元素周期表就可以看成是一个集合,每个化学元素都可以看成该集合的一个元素。
\begin{definition}(集合和元素)
    可以归为一类的东西,构成的整体称为集合,其中的每样东西称为该集合的元素。
\end{definition}

集合必须具有以下三个性质,否则就不是一个集合:
(1)确定性。集合中的每个元素都必须是确定的;
(2)互异性。集合中的每个元素都是不同的;
(3)无序性。集合中的每个元素都没有顺序。

\begin{example}
    以下哪些是集合?哪些不是?
    (1)所有偶数;
    (2)100米跑中用时小于5s的运动员;
    (3)100米跑中用时较短的运动员。
\end{example}

\begin{solution}
    (1)是集合。
    (2)也是集合。
    (3)不是集合。因为不具有确定性。
\end{solution}

集合一般有三种描述方法:
(1)第一种就是像上例(1)和(2)一样使用自然语言来描述;
(2)第二种就是列举出来,一般我们每个元素之间使用“,”来描述,再在最外层使用大括号包裹,例如$\{1,2,3\}$就表示一个含有3个元素的集合(由于集合的无序性,所以它和集合$\{3,1,2\}$等是相同的);
(3)描述法。简而言之,就是使用自然语言或者公式来描述集合中元素的共同特征,例如$\{x|x是偶数\}$就描述了一个集合,该集合含有的元素都是符合“该元素是偶数”这么一个特征的元素,即该集合表示偶数集合。

集合一般用大写字母表示,其中的元素一般用小写字母表示。
如果两个集合相同(含有的元素数量相同、且含有相同的元素、没有不同的元素),那么称两个集合相等。
例如$\{1,3,2\}=\{1,2,3\}$,等号两侧的集合是相等的。于是乎$A=\{1,2,3\}$的意思就是说说用大写字母$A$来表示右侧那个集合。

有一些集合,比较常用,我们用特别的字母来表示,在表示其它集合时,不应使用这些特定字母:
(1)$R$表示实数集,$C$表示复数集。本系列中,如果没有特殊说明,均在实数范围内探讨;
(2)$Z$表示整数集,$Z^+$表示正整数集,$Z^-$表示负整数集;
(3)$N$表示自然数集,由于不同地方对自然数的定义不同,因此应尽量避免使用
(4)$Q$表示有理数集。

\begin{definition}(集合和元素的关系)
    如果元素$a$在集合$A$中,称为元素$a$\textbf{属于}集合$A$,记为$a \in A$;
    如果元素$a$不在集合$A$中,称为元素$a$\textbf{不属于}集合$A$,极为$a \notin A$. 
\end{definition}

\subsection{集合与集合之间的关系}

\subsection{集合之间的运算}