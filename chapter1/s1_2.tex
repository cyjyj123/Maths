\section{集合}
\subsection{集合的概念和表示}
多姿多彩的世界有着多种多样的物质。无论是什么物质,都是由特定的化学元素构成的。
虽然每一种化学元素的具体性质都不同,但是无论是何种化学元素,我们都可以把它们归为一类——即“化学元素”。
我们把可以归为一类的对象组合在一起构成的总体称为\textbf{集合}.其中的一个对象称为该集合的一个\textbf{元素}.
例如,元素周期表就可以看成是一个集合,每个化学元素都可以看成该集合的一个元素。
\begin{definition}(集合和元素)
    可以归为一类的东西,构成的整体称为集合,其中的每样东西称为该集合的元素。
\end{definition}

集合必须具有以下三个性质,否则就不是一个集合:
(1)确定性。集合中的每个元素都必须是确定的;
(2)互异性。集合中的每个元素都是不同的;
(3)无序性。集合中的每个元素都没有顺序。

\begin{example}
    以下哪些是集合?哪些不是?
    (1)所有偶数;
    (2)100米跑中用时小于5s的运动员;
    (3)100米跑中用时较短的运动员。
\end{example}

\begin{solution}
    (1)是集合。
    (2)也是集合。
    (3)不是集合。因为不具有确定性。
\end{solution}

集合一般有三种描述方法:
(1)第一种就是像上例(1)和(2)一样使用自然语言来描述;
(2)第二种就是列举出来,一般我们每个元素之间使用“,”来描述,再在最外层使用大括号包裹,例如$\{1,2,3\}$就表示一个含有3个元素的集合(由于集合的无序性,所以它和集合$\{3,1,2\}$等是相同的);
(3)描述法。简而言之,就是使用自然语言或者公式来描述集合中元素的共同特征,例如$\{x|x$是偶数$\}$就描述了一个集合,该集合含有的元素都是符合“该元素是偶数”这么一个特征的元素,即该集合表示偶数集合。

集合一般用大写字母表示,其中的元素一般用小写字母表示。
如果两个集合相同(含有的元素数量相同、且含有相同的元素、没有不同的元素),那么称两个集合相等。
例如$\{1,3,2\}=\{1,2,3\}$,等号两侧的集合是相等的。于是乎$A=\{1,2,3\}$的意思就是说说用大写字母$A$来表示右侧那个集合。

有一些集合,比较常用,我们用特别的字母来表示,在表示其它集合时,不应使用这些特定字母:
(1)$R$表示实数集,$C$表示复数集。本系列中,如果没有特殊说明,均在实数范围内探讨;
(2)$Z$表示整数集,$Z^+$表示正整数集,$Z^-$表示负整数集;
(3)$N$表示自然数集,由于不同地方对自然数的定义不同,因此应尽量避免使用
(4)$Q$表示有理数集。

\begin{definition}(集合和元素的关系)
    如果元素$a$在集合$A$中,称为元素$a$\textbf{属于}集合$A$,记为$a \in A$;
    如果元素$a$不在集合$A$中,称为元素$a$\textbf{不属于}集合$A$,极为$a \notin A$. 
\end{definition}

集合中的元素个数称为该集合的\textbf{基数}。注意,由于集合中的元素是不同的,因此集合的基数指的是集合中不同元素的个数。

基数为0的集合,记为$\emptyset$.

\subsection{集合与集合之间的关系}
如果集合$A$中的所有元素,集合$B$也有,那么称$A$包含于$B$(或$B$包含$A$)。

显然这有两种情况,一种是$B$除了含有$A$中的元素还有其它元素,即它们不相等;
第二种情况就是$B$中只有$A$中的元素,而没有其它的元素,即它们是相等的。
无论哪种情况,我们都可以称$A$包含于$B$,只不过在它们不相等的时候,也可以称$A$真包含于$B$(或$B$真包含$A$)。

简而言之,如果$A$包含于$B$,它们可能相等也可能不等,但是如果$A$真包含于$B$,那么它们一定不等。

如果$A$包含于$B$,则称$A$是$B$的子集,特别地,空集$\emptyset$是任何集合的子集;
如果$A$真包含于$B$,则称$A$是$B$的真子集。显然一个集合的真子集一定是该集合的一个子集,但是该集合的一个子集不一定是该集合的一个真子集。

\begin{definition}(包含和子集的符号化定义)
    称$\forall x \in A \to x \in B $为$A$包含于$B$,$A$是$B$的子集,记为$A \subset B$;

    称$\forall x  \in A \to (x \in B \wedge A \neq B)$为$A$真包含于$B$,$A$是$B$的真子集,记为$A \subsetneqq B$。
\end{definition}

\subsection{集合之间的运算}

\subsection{区间}
为了方便,我们使用\textbf{区间}来表示一些特殊的、具有连续性的集合。
\begin{definition}
    $[a,b]$表示集合$\{x| a \leq  x \leq b\}$;
    
    $(a,b)$表示集合$\{x| a<x<b\}$;

    其中$a$和$b$称为该区间的\textbf{端点}.

    类似地,
    $[a,b)$表示集合$\{x| a \leq x <b\}$;
    $(a,b]$表示集合$\{x| a < x \leq b\}$.
\end{definition}
简而言之,我们将两个数字从小到大放入括号中,它所表示的集合是处于这两个数字之间的所有实数,这个集合中是否包含有端点则看括号的类型,
如果是中括号就包括,如果是小括号就不包括。区间的两个端点的括号类型可以是不同的,哪边是方括号就哪边包括,两边都是方括号就两边都包括;哪边是小括号哪边就不包括,两边都是小括号那么就是两边都不包括。

特殊地,我们可以使用正负无穷大符号来表达大于(大于等于)某个数、小于(小于等于)某个数。负无穷大一定在区间的左侧,正无穷大一定在区间的右侧。
\begin{definition}
    $(-\infty,a)$表示集合$\{x | x < a\}$;

    $(-\infty,a]$表示集合$\{x | x \leq a\}$;
    
    $(a,+\infty)$表示集合$\{x | x > a\}$;
    
    $[a,+\infty)$表示集合$\{x | x \geq a\}$;
    
    特别地,
    
    $(-\infty,+\infty)$表示实数集R.
\end{definition}
由于正无穷大和负无穷大不可能取到,所以正无穷大和负无穷大那侧只能是小括号。
注意,区间只是一些特殊的集合的简便记法,因此它的本质依然是集合,所以也能够进行集合之间的各种关系判断和运算。

\section*{习题 \thesection}