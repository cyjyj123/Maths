\documentclass[fontset=none]{ctexbook}
\usepackage{amsthm}
\usepackage{amsmath}
\usepackage{exerquiz}

\setCJKmainfont{Noto Serif CJK SC}

\ctexset{
    chapter={
        number={\arabic{chapter}}
    },
    section={
        number={\arabic{chapter}.\arabic{section}}
    },
    subsection={
        number={\arabic{chapter}.\arabic{section}.\arabic{subsection}}
    }
}



\newtheorem{theorem}{\indent 定理}[section]
\newtheorem{lemma}[theorem]{\indent 引理}
\newtheorem{proposition}[theorem]{\indent 命题}
\newtheorem{corollary}[theorem]{\indent 推论}
\newtheorem{definition}{\indent 定义}[section]
\newtheorem{example}{\indent 例}[section]
\newtheorem{remark}{\indent 注}[section]
\newenvironment{solution}{\begin{proof}[\indent\bf 解]}{\end{proof}}
\renewcommand{\proofname}{\indent\bf 证}



\begin{document}

\sqTurnOffAlerts
{\Huge Maths \par \vspace{1.5em} 数学}

\rule{\textwidth}{1pt}

{\Huge\rightline{Level 1}}

\begin{center}
    \Large \textbf{前言}
\end{center}

我们很高兴可以开启本项目,完成本项目的第一个版本的时间可能也会很长。欢迎大家来一起学习数学和贡献内容。

本项目主要内容包括适用于工科的初等数学、微积分、线性代数、概率论和数理统计等内容,主要会分为5册,其内容如分册目录所述,每册内容可能会随时改变。如果有足够的时间,一年学习1册是比较合适的,这也是分册目录存在的意义之一。
此项目为开源内容,您可以在Github上找到,在此项目中,许多人为此贡献过内容,您可以在本项目的authors$.$md文件中找到部分贡献者,也有部分贡献者不愿在该文件中写入他们的名字,这里一并致谢!

本系列内容可能有所谬误,敬请指正。

\begin{center}
    \Large \textbf{使用说明}
\end{center}

本系列主要使用Latex编写,如果您使用现代计算机阅读此文档,则可以进行交互。如果需要进行交互,您必须使用支持JavaScript的PDF阅览器,例如使用Edge浏览器打开pdf文件。

其中,练习部分主要使用Acrotex,下面是一些交互的示例,您可以点击选项。

\begin{shortquiz*}
    请选择唯一正确的答案,选择后,选框会反馈您选择的正确性。
    \begin{answers}{4}
        \bChoices
        \Ans0 这是错误的答案 \eAns
        \Ans1 这是正确的答案 \eAns
        \eChoices
    \end{answers}
\end{shortquiz*}

\begin{shortquiz*}
    这是多选
    \begin{manswers}{5}
        \Ans0 错误 &
        \Ans1 正确 &
        \Ans1 正确 &
        \Ans0 错误 &
        \Ans1 正确
    \end{manswers}
\end{shortquiz*}

{\tableofcontents}

\chapter{命题逻辑和集合}
\section{命题}
\subsection{命题的概念和定义}
我们在生活中,经常会遇到许多句子,其中有陈述句、感叹句、疑问句等。
在这些句子中,有一类句子可以判断其是否符合事实,这类句子一定是陈述句。
如果它符合事实那么就称它为真,如果不符合事实就称它为假。
我们把这类可以判断真假的句子叫做\textbf{命题}.如果它为真,则称为\textbf{真命题},如果它为假,则称为\textbf{假命题}.命题的真假称为该命题的\textbf{真值}.

\begin{definition}[命题]
    能够判断真假的陈述句。
\end{definition}

虽然命题一定是陈述句,但是不能判断真假的陈述句不是命题,所以陈述句不一定是命题。

真命题的真值一般记为T、True、1,假命题的真值一般记为F、False、0.不过在数学上,我们一般习惯使用数字,即记为1和0.

为了符号化命题,我们引入命题常元和命题变元的概念,
它们一般均用小写字母$p,q...$等表示。

\begin{definition}
    只能取值为真或假的变量,称为\textbf{命题变元}。
    (亦称\textbf{命题变项})
    
    值为真或假的常量称为\textbf{命题常元}。(亦称\textbf{命题常项})
\end{definition}

例如,有一变量$p$只能取值为真或假,那么它就是命题变元。
而如果是这样的:“$p$:太阳在自转”,那么此时$p$就是一个命题常项。

从上述例子可以看出,为命题变元赋值一个命题之后,它就变成了一个命题常元。
因此命题常元代表着一个命题,而命题变元不是命题。

有时候,我们并不关心具体的命题,因此我们直接使用$1$和$0$来分别代表真命题和假命题,
它们也是命题常项。

\begin{example}
    假设昨天下雨了,判断以下句子是否是命题,如果是,则是真命题还是假命题?

    (1)昨天下雨了吗?

    (2)昨天下雨了。
    
    (3)昨天没下雨。
\end{example}
\begin{solution}
    (1)是疑问句,所以不是命题;
    
    (2)是真值为真的陈述句,所以是真命题;

    (3)是真值为假的陈述句,所以是假命题。
\end{solution}

等式、不等式这类式子可以视为陈述句,所以只要它能够判断真假(也就是该式子是否成立)就可以算作命题。
如果该式子成立,那么就是真命题,否则就是假命题。

\begin{example}
    判断下列命题的真值:

    (1)直角三角形斜边长度的平方等于其两直角边长度的平方之和。
    
    (2)$1+1>2$

    (3)$1+1=2$
\end{example}
\begin{solution}
    (1)就是毕达哥拉斯定理(勾股定理),是真命题,它的真值为真;
    
    (2)该不等式不成立,所以是假命题,真值为假;
    
    (3)该等式成立,所以是真命题,其真值为真。
\end{solution}

    判断命题为假的方式只要举出一个反例即可,而判断命题为真,要么它是一个公理,要么通过证明。证明的方式多种多样,我们会在之后进行叙述。

\subsection{逻辑联结词}
一些命题是简单的,我们把它们称为原子命题,一些命题是复杂的,但可以把它们拆分为若干个原子命题,并用一些符号连接起来,这些符号称为\textbf{逻辑联结词},我们把这些复杂的命题称为\textbf{复合命题}。
简而言之,复合命题是由若干个原子命题通过逻辑联结词连接起来构成的。

常用的逻辑联结词有\textbf{蕴含}、\textbf{合取}、\textbf{析取}、\textbf{否定}和\textbf{双条件}。

\begin{definition}[蕴含]
    $p$蕴含$q$,记为$p \to q$,也可读作如果$p$、那么$q$,当且仅当$p$为真、$q$为假时才为假.
\end{definition}

当命题变元$p$和$q$取不同值时,该复合命题的真值如下表所示:

\begin{tabular}{c|c|c}
    $p$ & $q$ & $p \to q$ \\
    \hline
    0 & 0 & 1 \\
    0 & 1 & 1 \\
    1 & 0 & 0 \\
    1 & 1 & 1
\end{tabular}

\begin{definition}[合取]
    $p$合取$q$记为$p \wedge q$,当且仅当命题$p$和$q$均为真时才为真.
\end{definition}
\begin{tabular}{c|c|c}
    $p$ & $q$ & $p \wedge q$ \\
    \hline
    0 & 0 & 0 \\
    0 & 1 & 0 \\
    1 & 0 & 0 \\
    1 & 1 & 1
\end{tabular}

\begin{definition}[析取]
    $p$析取$q$记为$p \vee q$,当且仅当命题$p$和$q$均为假时才为假.
\end{definition}
\begin{tabular}{c|c|c}
    $p$ & $q$ & $p \vee q$ \\
    \hline
    0 & 0 & 0 \\
    0 & 1 & 1 \\
    1 & 0 & 1 \\
    1 & 1 & 1
\end{tabular}

\begin{definition}[否定]
    和命题变元$p$真值相反的命题变元,记为$\neg p$.
\end{definition}
\begin{tabular}{c|c|c}
    $p$ & $\neg p$  \\
    \hline
    0 & 1  \\
    1 & 0 
\end{tabular}

\begin{definition}[双条件]
    命题变元$p$和$q$的真值相同时为真、不同时为假,记为$p \leftrightarrow q$.
\end{definition}
\begin{tabular}{c|c|c}
    $p$ & $q$ & $p \leftrightarrow q$ \\
    \hline
    0 & 0 & 1 \\
    0 & 1 & 0 \\
    1 & 0 & 0 \\
    1 & 1 & 1
\end{tabular}

像上面这样,写出各命题所有取值情况时的复合命题的真值的表格,称为\textbf{真值表}.

\begin{example}
    设有命题$p$:太阳在自转,$q$:地球在自转,
    那么:

    (1)$p \to q$所表达的命题为:如果太阳在自转,那么地球在自转。\par
    由于这两个命题都是真命题,所以该复合命题是真命题,虽然太阳是否自转和地球是否自转没有关系。

    (2)$p \wedge q$所表达的命题为:太阳在自转\textbf{且}地球在自转。\par
    这是一个真命题。

    (3)$p \vee q$所表达的命题为:太阳在自转\textbf{或}地球在自转。\par
    这是一个真命题。

    (4)$\neg q$表示:地球不在自转。这显然是一个假命题。

    (5)$p \wedge \neg q$表示:太阳在自转且地球不在自转。\par
    由于只有两个命题均为真时,合取命题才为真,这里左边的命题为真,右边的命题为假,所以该复合命题是一个假命题。

    (6)$\neg p \vee \neg q$表示:太阳不在自转或地球不在自转。\par
    由于两边都是假命题,根据析取的定义,该复合命题是假命题.
\end{example}

在上例中,$p$和$q$为命题常项(因为为它们赋值了,因此它们都是命题),所以由它们和联结词构成的是一个复合命题。
如果我们不为$p$和$q$赋值,使它们成为命题变元,那么它们和联结词构成的就是一个真值未定的式子,此时就不是一个命题,我们把这样的式子叫做\textbf{命题公式}(简称为公式),当命题变元的真值确定后,该公式的真值随即确定,既然真值确定了,那么它也就变成了一个命题。

联结词除了能用于命题常元和命题变元直接的运算,也能用于合式公式之间的运算,因为只要参加运算的项确定了真值就可以通过联结词进行运算,而合式公式也具有真值,因此也能进行联结词运算。

一般我们用大写字母表示命题公式。我们还可以使用小括号表示该公式中含有的命题变元,但这不是必须的。例如$A$可以表示一个公式,$A(p,q)$则表示该公式含有命题变元$p$和$q$.
命题常元和命题变元本身也是一个命题公式,所以公式中可能只有1个命题变元,也可能像上面一样,有2个命题变元,也可能有3个等等。

命题公式由命题变元、联结词和括号构成,从左到右运算,先算括号内再算括号外。除此之外,联结词的优先级从高到低为否定、合取、析取、蕴含、双条件。

命题公式是递归定义的,它的具体生成规则如下:
\begin{definition}[命题公式]
    又称为\textbf{合式公式},一般用大写字母表示

    (1)命题常元和命题变元是命题公式;

    (2)如果$A$是命题公式,那么$\neg A$是命题公式;

    (3)如果$A$和$B$是命题公式,那么$(A \wedge B)$、$(A \vee B)$、$(A \to B)$、$(A \leftrightarrow B)$都是命题公式;
    
    (4)有限次地应用上述步骤得到的也是命题公式。
\end{definition}

通俗地讲,一个命题公式中有若干个命题变元,这些变元在没有赋值的情况下,由于不是一个命题,所以此时该命题公式不是一个命题。而当每个命题变元都有了值后,就变成了命题,也就相当于把这些值代入到了命题公式中,于是形成了一个复杂命题。
当该命题公式中的各个命题变元取一个值时,称为该命题公式的一个\textbf{真值指派}(简称指派);
命题公式的一个指派和此时该公式的真值称为该命题公式的一个\textbf{解释}。
由于每个命题变元只可能取2个值,所以对于具有$n$个命题变元的命题公式,
有$2^n$种取法,因此有$2^n$种指派,$2^n$种解释。

\begin{example}
    设有命题变元$p,q,r$判断下列是否是命题公式?
    (1)$p$;
    (2)$p \wedge q$;
    (3)$p \neg \vee q \to r$;
    (4)$p \vee \neq q \to r$;
    (5)$(p \vee \neq q) \to r$;
\end{example}
\begin{solution}
    (1)是公式。因为命题变元本身也是命题公式;
    (2)是公式。
    (3)不是公式。因为$\neg$联结词只能在命题变元的前面;
    (4)是公式。顺便说一句,此时,先算$\neq q$,然后算$\neq \to r$,最后再算$p$析取上一步的结果;
    (5)是公式。和(4)的区别在于先算括号内的,然后在计算结果蕴含$r$.
\end{solution}

命题公式的真值情况也可以使用真值表表示。只要在表格中列出命题公式中含有的所有命题变元和所有指派,并计算每种指派情况下公式的真值即可。

如果两个命题公式$A$和$B$在所有的指派下,其对应的真值总是相同的,那么称它们等价,记为$A \Leftrightarrow B$.
于是,我们可以推导出一些规律,例如$\neg \neg A \Leftrightarrow A$,从而简化后面的计算。

\begin{theorem}(双重否定)
    设有命题公式$A$,则$\neg \neg A \Leftrightarrow A$.
\end{theorem}
\begin{proof}
    列表即证:
    \begin{tabular}{c|c|c}
        $A$ & $\neg A$ & $\neg \neg A$ \\
        \hline
        0 & 1 & 0 \\
        1 & 0 & 1 \\
    \end{tabular}

    我们主要看第一列和第三列,无论$A$取$0$还是$1$,$\neg \neg A$都是对应相等的,因此这两者是等价的。
\end{proof}

主要的运算规律:

(1)交换律:

$A \wedge B \Leftrightarrow B \wedge A$

$A \vee B \Leftrightarrow B \vee A$

% 还有一些之后再写

当然命题变元也是命题公式,所以也是遵循上面的运算规律的。

命题公式主要可以分为三类:
\begin{definition}(永真式)
    设命题公式$A(p,q,...)$,无论$p,q...$取什么值,命题公式A总为真,该命题公式称为\textbf{永真式},又称为\textbf{重言式}.
\end{definition}
\begin{definition}(永假式)
    设命题公式$A(p,q,...)$,无论$p,q...$取什么值,命题公式A总为假,该命题公式称为\textbf{永假式},又称为\textbf{矛盾式}.
\end{definition}
\begin{definition}
    设命题公式$A(p,q,...)$,只要$p,q...$有一组指派使得命题公式A为真,该命题公式称为\textbf{可满足式}.
\end{definition}

显然,永真式一定是可满足式。

根据上述定义,如果有命题公式$A$,且$A \Leftrightarrow 1$,
那么就意味着无论$A$无论取何种指派,其真值都为1,即$A$是永真式;
同理,$A \Leftrightarrow 0$表示$A$是永假式。

\subsection{命题的逆命题、否命题和逆否命题}
有了逻辑联结词,我们很容易可以讲解这三个概念。 
\begin{definition}[逆命题]
    设有命题$p \to q$,其逆命题为$q \to p$.
\end{definition}
\begin{definition}[否命题]
    设有命题$p \to q$,其否命题为$\neg p \to \neg q$.
\end{definition}
\begin{definition}[逆否命题]
    设有命题$p \to q$,其逆否命题为$\neg q \to \neg p$.
\end{definition}
这里,重要的一点是一个命题的逆否命题和该命题的真值相同。在证明它之前,我们先引入几个概念:



\begin{theorem}
    $p \to q \iff \neg q \to \neg p$,即命题$p \to q \leftrightarrow \neg q \to \neg p$是永真式(也就是该式$\Leftrightarrow 1$).
\end{theorem}
\begin{proof}
    (1)第一种方法:列表即证:

    \begin{tabular}{c|c|c|c|c|c}
        $p$ & $q$ & $\neg q$ & $\neg p$ & $p \to q$ & $\neg q \to \neg p$ \\
        \hline
        0 & 0 & 1 & 1 & 1 & 1 \\
        0 & 1 & 0 & 1 & 1 & 1 \\
        1 & 0 & 1 & 0 & 0 & 0 \\
        1 & 1 & 0 & 0 & 1 & 1
    \end{tabular}

    (2)第二种方法:
    证明:$p \to q \leftrightarrow \neg q \to \neg p \Leftrightarrow 1$.

    计算等价号$\Leftrightarrow$左侧:
    $p \to q \leftrightarrow \neg q \to \neg p\Leftrightarrow ((p \to q) \to (\neg q \to \neg p)) \vee ((\neg q \to \neg p) \to (p \to q))$,

    该式子根据析取号分为两项,只要任意一项等价于$1$,该式子就等价于$1$,
    我们计算左边这一项:
    \begin{equation*}
        \begin{aligned}
    & (p \to q) \to (\neg q \to \neg p) \\
    & \Leftrightarrow (\neg (p \to q) \vee (\neg q \to \neg p)) \\
    & \Leftrightarrow \neg (\neg p \vee q) \vee (\neg \neg q \vee \neg p) \\
    & \Leftrightarrow p \wedge \neg q \vee q \vee \neg p  \\
    & \Leftrightarrow p \wedge \neg q \vee \neg p \vee q \\
    & \Leftrightarrow  (p \vee \neg p) \wedge (\neg q \vee \neg p) \vee q \\
    & \Leftrightarrow 1 \wedge (\neg q \vee \neg p) \vee q \\
    & \Leftrightarrow \neg q \vee \neg p \vee q \\
    & \Leftrightarrow \neg q \vee q \vee \neg p \\
    & \Leftrightarrow 1 \vee \neg p \\
    & \Leftrightarrow 1
        \end{aligned}
    \end{equation*}

    即证。
\end{proof}

\subsection{必要条件、充分条件、充分必要条件}
\begin{definition}
    如果命题$p$可以推出命题$q$,则记为$p \Rightarrow q $.

    其中,$p$称为$q$的\textbf{充分条件},$q$称为$p$的必要条件。
\end{definition}

\begin{definition}
    有命题$p,q$,如果$p \Rightarrow q$且$q \Rightarrow p$(即$(p \Rightarrow q) \wedge (q \Rightarrow p)$),
    则记为$p \Leftrightarrow q$.

    称$p$是$q$的\textbf{充分必要条件},简称\textbf{充要条件},
    显然,$q$也是$p$的充要条件。
\end{definition}

充要条件的符号和等价符号是一样的,因为它们本质是一回事。

\begin{example}
    判断下列命题$p$是$q$的什么条件?

    (1)
\end{example}

\subsection{量词}
设有命题“$p$:对于任何实数$x$,$2x$是偶数”,
我们可以得到它的否命题“$\neg p$:存在实数$x$,使得$2x$是奇数”。

上面,我们在命题的条件中描述$x$时用到了一些限定词:“任何”、“存在”。
这些词称为\textbf{量词}.
表示“任何的、所有的”的量词称为\textbf{全称量词},记为$\forall$;
表示“至少存在1个”的量词称为 \textbf{存在量词},记为$\exists$。

我们先引入一个概念,“$x \in R$”表示“$x$是一个实数”。具体的含义会在下一节中说明。

量词一般写在字母前面,例如“对于任何实数$x$”可以写成“$\forall x \in R$”,“存在一个实数$x$”可以写成“$\exists x \in R$”.

\section*{习题 \thesection}

\section{集合}
\subsection{集合的概念和表示}
多姿多彩的世界有着多种多样的物质。无论是什么物质,都是由特定的化学元素构成的。
虽然每一种化学元素的具体性质都不同,但是无论是何种化学元素,我们都可以把它们归为一类——即“化学元素”。
我们把可以归为一类的对象组合在一起构成的总体称为\textbf{集合}.其中的一个对象称为该集合的一个\textbf{元素}.
例如,元素周期表就可以看成是一个集合,每个化学元素都可以看成该集合的一个元素。
\begin{definition}(集合和元素)
    可以归为一类的东西,构成的整体称为集合,其中的每样东西称为该集合的元素。
\end{definition}

集合必须具有以下三个性质,否则就不是一个集合:
(1)确定性。集合中的每个元素都必须是确定的;
(2)互异性。集合中的每个元素都是不同的;
(3)无序性。集合中的每个元素都没有顺序。

\begin{example}
    以下哪些是集合?哪些不是?
    (1)所有偶数;
    (2)100米跑中用时小于5s的运动员;
    (3)100米跑中用时较短的运动员。
\end{example}

\begin{solution}
    (1)是集合。
    (2)也是集合。
    (3)不是集合。因为不具有确定性。
\end{solution}

集合一般有三种描述方法:
(1)第一种就是像上例(1)和(2)一样使用自然语言来描述;
(2)第二种就是列举出来,一般我们每个元素之间使用“,”来描述,再在最外层使用大括号包裹,例如$\{1,2,3\}$就表示一个含有3个元素的集合(由于集合的无序性,所以它和集合$\{3,1,2\}$等是相同的);
(3)描述法。简而言之,就是使用自然语言或者公式来描述集合中元素的共同特征,例如$\{x|x是偶数\}$就描述了一个集合,该集合含有的元素都是符合“该元素是偶数”这么一个特征的元素,即该集合表示偶数集合。

集合一般用大写字母表示,其中的元素一般用小写字母表示。
如果两个集合相同(含有的元素数量相同、且含有相同的元素、没有不同的元素),那么称两个集合相等。
例如$\{1,3,2\}=\{1,2,3\}$,等号两侧的集合是相等的。于是乎$A=\{1,2,3\}$的意思就是说说用大写字母$A$来表示右侧那个集合。

有一些集合,比较常用,我们用特别的字母来表示,在表示其它集合时,不应使用这些特定字母:
(1)$R$表示实数集,$C$表示复数集。本系列中,如果没有特殊说明,均在实数范围内探讨;
(2)$Z$表示整数集,$Z^+$表示正整数集,$Z^-$表示负整数集;
(3)$N$表示自然数集,由于不同地方对自然数的定义不同,因此应尽量避免使用
(4)$Q$表示有理数集。

\begin{definition}(集合和元素的关系)
    如果元素$a$在集合$A$中,称为元素$a$\textbf{属于}集合$A$,记为$a \in A$;
    如果元素$a$不在集合$A$中,称为元素$a$\textbf{不属于}集合$A$,极为$a \notin A$. 
\end{definition}

\subsection{集合与集合之间的关系}

\subsection{集合之间的运算}
\section{集合论在概率论中的应用}

\chapter{方程(组)和不等式}
\section{矩阵和线性方程组}
本节将假定您已经掌握了一些基本的解方程(组)和不等式(组)的知识(相当于初中毕业),我们首先将会扩展方程和方程组的一些概念,并引入矩阵来简写一些方程组的解法。
\subsection{线性方程组}
\begin{definition}
    如果一个方程可以化为$a_1x_1+a_2x_2+...+a_nx_n=c$(其中$a_1,a_2,...a_n,c$为常数,$x_1,x_2,...,x_n$为未知数)的形式,那么该方程称为\textbf{线性方程}。

    特别地,如果$c$为0,称该方程为\textbf{齐次线性方程},如果$c$不为0,那么称为\textbf{非齐次线性方程}.
\end{definition}
简而言之,一个方程如果由含有未知数的项的多项式和一个常数构成,且未知数的次数都为1,那么这个方程可以被称作线性方程。否则就不是线性方程,称为\textbf{非线性方程}。

线性方程之所以称为线性方程是因为如果把它变成函数放到坐标系中,它的形状形似一条直线。

\begin{example}
方程$3x-2y+3=0$是线性方程,因为含有未知数的项的未知数的次数均为1;
方程$3x^2-2y+3=0$是非线性方程,因为未知数$x$的次数不为1.
\end{example}

一个方程组是由若干个方程构成的,如果方程组中的所有方程都是线性方程,那么称该方程组为\textbf{线性方程组},如果该方程组有一个方程不是线性方程,那么它就不是线性方程组,可以称其为非线性方程组。同理,如果线性方程组中的所有方程都是齐次方程(因为是在线性方程组中,所以一定是齐次线性方程),那么该方程组称为\textbf{齐次线性方程组},但是只要线性方程组中有一个方程是非齐次线性方程,那么该方程组就是\textbf{非齐次线性方程组}。

\begin{example}
    \begin{equation}
        \begin{cases}
            2x+3y=0 \\
            5x-y=0
        \end{cases}
    \end{equation}
    是齐次线性方程组;

    \begin{equation}
        \begin{cases}
            2x+3y=8 \\
            5x-y=0
        \end{cases}
    \end{equation}
    是非齐次线性方程组。
\end{example}

\subsection{矩阵}
\begin{definition}
    将若干个数排成$m$行$n$列的数表,并用括号括起来称为$m \times n$型\textbf{矩阵},一般用大写字母表示记为$A_{mn}$,如果可以忽略行列,则可以简记为$A$.
    
    若记矩阵$m \times n$型矩阵$A$第$i$行第$j$列的元素为$a_{ij}$,则矩阵$A$可以记为:

    $A=\begin{pmatrix} a_{11} & a_{12} & ... & a_{1n} \\ a_{21} & a_{22} & ... & a_{2n} \\ ...  \\ a_{m1} & a_{m2} & ... &a_{mn} \end{pmatrix}$

    特别地,如果$m=n$,那么称该矩阵为$m$阶\textbf{方阵}.
\end{definition}

在一些文档中,矩阵的括号也可能会写成方括号。

矩阵的行数和列数可以是不同的。如果两个矩阵的行数相等,并且列数也相等则称它们为\textbf{同型矩阵},只有同型矩阵才能比较。如果两个同型矩阵对应行数和列数的元素都相等,那么称这两个矩阵相等。

有了矩阵的概念之后,我们就可以更简便地表示线性方程组。我们先把线性方程组中每个方程含有未知数的项放在等号左边,常数项放在等号右边。左边的未知数排号顺序,例如第一个方程如果按$x,y,...$排序,后面的方程也要按照这个顺序。
我们将整理好的线性方程组每一个方程左侧每一项(即含有未知数的项)的系数按顺序从左到右排成矩阵,该矩阵称为该线性方程组的\textbf{系数矩阵}。

\begin{example}
    线性方程组
    \begin{equation}
        \begin{cases}
            3x+2y=5 \\
            10x-8y=11
        \end{cases}
    \end{equation}
    的系数矩阵是
    $\begin{pmatrix}
        3 & 2 \\ 10 & -8
    \end{pmatrix}$    
\end{example}

如果将右侧的常数也放到矩阵中,那么称该矩阵为该线性方程组的\textbf{增广矩阵}.

\begin{example}
    以上例,其增广矩阵为

    $\begin{pmatrix}
        3 & 2 & 5 \\ 10 & -8 & 11 
    \end{pmatrix}$   
\end{example}

\begin{definition}
    $n$阶方阵中,$a_{11},a_{12},...,a_{nn}$连线称为该方阵的主对角线;$a_{1n},a_{2,n-1},a_{n1}$的连线称为副对角线。
    如果一个方阵的所有元素都为$0$,那么称为\textbf{零矩阵},记为$O$;
    如果一个方阵的主对角线的元素都为$1$,且其它元素都为$0$,那么称为\textbf{单位阵},记为$E$.
\end{definition}

\subsection{解线性方程组}
在以前,我们解线性方程组使用的方法称为\textbf{高斯消元法},其主要操作包括三种:

(1)交换两个方程;

(2)在一个方程上乘上某个非零数;

(3)在一个方程上乘上某个非零数加到另一个方程。

每步操作形成的方程均为同解方程,因此最终我们能够求得原方程组的解。
如果把每步操作所形成的新方程写成矩阵的形式,那么原来的方程到新的方程可以转换成原矩阵变成新矩阵。
我们将这种原矩阵到新矩阵的操作称为矩阵的\textbf{初等变换}。

初等变换可以在行上进行,列也是一样的,分别称之为\textbf{初等行变换}和\textbf{初等列变换}。
我们以初等行变换为例,原来的三种操作可以分别变成:

(1)交换矩阵中的两行;

(2)在矩阵的某行乘上一个非零数;

(3)在矩阵的某行乘上一个非零数加到另一行上。

那么我们就可以通过不断地变换增广矩阵来求解线性方程组。

例如,对于增广矩阵为$\begin{pmatrix}
    3 & 2 & a \\ 10 & -8 & b 
\end{pmatrix}$的线性方程组,如果其矩阵可以变换为
$\begin{pmatrix}
    1 & 0 & c \\ 0 & 1 & d 
\end{pmatrix}$,那么我们就容易求得原方程组的两个未知数的解分别为$c$和$d$.

简而言之,求解线性方程组的步骤就是将其增广矩阵通过初等变换变成除最后一列之外只含有0和1的矩阵即可(因为这样子的矩阵所对应的方程,一般情况下,每个方程的左侧都是一个系数为1的未知数,其它未知数的系数都变成0了,那么该未知数自然就等于右侧那个常数;然后每行中1所在的列不同,那么对应的就是不同的未知数了)。

初等变换一般记为$\sim$,它不是等于,初等变换不是等于号的含义,它们是不同的。矩阵变换后,和原矩阵也是同型矩阵,但是两个同型矩阵相等必须每个元素都相同,而矩阵变换的含义是它们对应的线性方程组同解,变换之后的矩阵和原来的矩阵的对应元素不一定相等,因此变换后的矩阵不一定等于原来的矩阵。

\begin{example}
    使用初等变换解线性方程组
    \begin{equation}
        \begin{cases}
            3x+2y=5 \\
            10x-8y=11
        \end{cases}
    \end{equation}
\end{example}
\begin{solution}
    设$A$为该线性方程组的增广矩阵,
    则
    $A= \begin{pmatrix}
        3 & 2 & 5 \\ 
        10 & -8 & 11 
    \end{pmatrix}$,

    $A \sim
    \begin{pmatrix}
        \frac{3}{2} & 1 & \frac{5}{2} \\ 
        -\frac{5}{4} & 1 & -\frac{11}{8} 
    \end{pmatrix}
    \sim
    \begin{pmatrix}
        \frac{11}{4} & 0 & \frac{31}{8} \\ 
        -\frac{5}{4} & 1 & -\frac{11}{8} 
    \end{pmatrix}
    \sim
    \begin{pmatrix}
        1 & 0 & \frac{31}{22} \\ 
        -\frac{5}{4} & 1 & -\frac{11}{8} 
    \end{pmatrix}
    \sim
    \begin{pmatrix}
        1 & 0 & \frac{31}{22} \\ 
        0 & 1 & \frac{17}{44} 
    \end{pmatrix}
    $,
    
    所以,原方程组的解为
    \begin{equation}
        \begin{cases}
            x=\frac{31}{22} \\
            y=\frac{17}{44}
        \end{cases}
    \end{equation}
\end{solution}

上例中,最后一个矩阵称之为\textbf{行最简型矩阵},因此求解线性方程组就是将对应的增广矩阵通过初等变换变成行最简型矩阵。

行最简型矩阵就是,它是\textbf{行阶梯型矩阵}的特例。行阶梯型矩阵就是

\subsection{未知数和方程数量不同的线性方程组}
之前我们所讨论的是未知数和方程数量相同的线性方程组(这里相同指的是每个方程都不同,也就是一个方程不能变成另一个方程,例如“x=2”和“3x=6”算相同的方程),
很明显,它有且仅有一个解($x,y,...$只是这个解的分量,它还是这个解的一部分,因此是\textbf{一个}解,而不是\textbf{一组}解,只有多个解才称为一组解。
\section{不等式的性质和基本不等式}
\section{一元二次方程和不等式}
\section{分式不等式与含绝对值的不等式}
\section{二元二次方程(组)和不等式(组)}
\chapter{幂、指数、对数及其函数}
\section{幂、指数和对数}
\section{幂函数、指数函数、对数函数与函数的基本性质和反函数}
\section{含有幂、指数和对数的方程(组)和不等式(组)}
\chapter{三角学}
\chapter{平面向量}
\chapter{复数}
\end{document}
