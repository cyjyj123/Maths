\section{矩阵和线性方程组}
本节将假定您已经掌握了一些基本的解方程(组)和不等式(组)的知识(相当于初中毕业),我们首先将会扩展方程和方程组的一些概念,并引入矩阵来简写一些方程组的解法。
\subsection{线性方程组}
\begin{definition}
    如果一个方程可以化为$a_1x_1+a_2x_2+...+a_nx_n=c$(其中$a_1,a_2,...a_n,c$为常数,$x_1,x_2,...,x_n$为未知数)的形式,那么该方程称为\textbf{线性方程}。

    特别地,如果$c$为0,称该方程为\textbf{齐次线性方程},如果$c$不为0,那么称为\textbf{非齐次线性方程}.
\end{definition}
简而言之,一个方程如果由含有未知数的项的多项式和一个常数构成,且未知数的次数都为1,那么这个方程可以被称作线性方程。否则就不是线性方程,称为\textbf{非线性方程}。

线性方程之所以称为线性方程是因为如果把它变成函数放到坐标系中,它的形状形似一条直线。

\begin{example}
方程$3x-2y+3=0$是线性方程,因为含有未知数的项的未知数的次数均为1;
方程$3x^2-2y+3=0$是非线性方程,因为未知数$x$的次数不为1.
\end{example}

一个方程组是由若干个方程构成的,如果方程组中的所有方程都是线性方程,那么称该方程组为\textbf{线性方程组},如果该方程组有一个方程不是线性方程,那么它就不是线性方程组,可以称其为非线性方程组。同理,如果线性方程组中的所有方程都是齐次方程(因为是在线性方程组中,所以一定是齐次线性方程),那么该方程组称为\textbf{齐次线性方程组},但是只要线性方程组中有一个方程是非齐次线性方程,那么该方程组就是\textbf{非齐次线性方程组}。

\begin{example}
    \begin{equation}
        \begin{cases}
            2x+3y=0 \\
            5x-y=0
        \end{cases}
    \end{equation}
    是齐次线性方程组;

    \begin{equation}
        \begin{cases}
            2x+3y=8 \\
            5x-y=0
        \end{cases}
    \end{equation}
    是非齐次线性方程组。
\end{example}

\subsection{矩阵}
\begin{definition}
    将若干个数排成$m$行$n$列的数表,并用括号括起来称为$m \times n$型\textbf{矩阵},一般用大写字母表示记为$A_{mn}$,如果可以忽略行列,则可以简记为$A$.
    
    若记矩阵$m \times n$型矩阵$A$第$i$行第$j$列的元素为$a_{ij}$,则矩阵$A$可以记为:

    $A=\begin{pmatrix} a_{11} & a_{12} & ... & a_{1n} \\ a_{21} & a_{22} & ... & a_{2n} \\ ...  \\ a_{m1} & a_{m2} & ... &a_{mn} \end{pmatrix}$

    特别地,如果$m=n$,那么称该矩阵为$m$阶\textbf{方阵}.
\end{definition}

在一些文档中,矩阵的括号也可能会写成方括号。

矩阵的行数和列数可以是不同的。如果两个矩阵的行数相等,并且列数也相等则称它们为\textbf{同型矩阵},只有同型矩阵才能比较。如果两个同型矩阵对应行数和列数的元素都相等,那么称这两个矩阵相等。

有了矩阵的概念之后,我们就可以更简便地表示线性方程组。我们先把线性方程组中每个方程含有未知数的项放在等号左边,常数项放在等号右边。左边的未知数排号顺序,例如第一个方程如果按$x,y,...$排序,后面的方程也要按照这个顺序。
我们将整理好的线性方程组每一个方程左侧每一项(即含有未知数的项)的系数按顺序从左到右排成矩阵,该矩阵称为该线性方程组的\textbf{系数矩阵}。

\begin{example}
    线性方程组
    \begin{equation}
        \begin{cases}
            3x+2y=5 \\
            10x-8y=11
        \end{cases}
    \end{equation}
    的系数矩阵是
    $\begin{pmatrix}
        3 & 2 \\ 10 & -8
    \end{pmatrix}$    
\end{example}

如果将右侧的常数也放到矩阵中,那么称该矩阵为该线性方程组的\textbf{增广矩阵}.

\begin{example}
    以上例,其增广矩阵为

    $\begin{pmatrix}
        3 & 2 & 5 \\ 10 & -8 & 11 
    \end{pmatrix}$   
\end{example}

\begin{definition}
    $n$阶方阵中,$a_{11},a_{12},...,a_{nn}$连线称为该方阵的主对角线;$a_{1n},a_{2,n-1},a_{n1}$的连线称为副对角线。
    如果一个方阵的所有元素都为$0$,那么称为\textbf{零矩阵},记为$O$;
    如果一个方阵的主对角线的元素都为$1$,且其它元素都为$0$,那么称为\textbf{单位阵},记为$E$.
\end{definition}

\subsection{解线性方程组}
在以前,我们解线性方程组使用的方法称为\textbf{高斯消元法},其主要操作包括三种:

(1)交换两个方程;

(2)在一个方程上乘上某个非零数;

(3)在一个方程上乘上某个非零数加到另一个方程。

每步操作形成的方程均为同解方程,因此最终我们能够求得原方程组的解。
如果把每步操作所形成的新方程写成矩阵的形式,那么原来的方程到新的方程可以转换成原矩阵变成新矩阵。
我们将这种原矩阵到新矩阵的操作称为矩阵的\textbf{初等变换}。

初等变换可以在行上进行,列也是一样的,分别称之为\textbf{初等行变换}和\textbf{初等列变换}。
我们以初等行变换为例,原来的三种操作可以分别变成:

(1)交换矩阵中的两行;

(2)在矩阵的某行乘上一个非零数;

(3)在矩阵的某行乘上一个非零数加到另一行上。

那么我们就可以通过不断地变换增广矩阵来求解线性方程组。

例如,对于增广矩阵为$\begin{pmatrix}
    3 & 2 & a \\ 10 & -8 & b 
\end{pmatrix}$的线性方程组,如果其矩阵可以变换为
$\begin{pmatrix}
    1 & 0 & c \\ 0 & 1 & d 
\end{pmatrix}$,那么我们就容易求得原方程组的两个未知数的解分别为$c$和$d$.

简而言之,求解线性方程组的步骤就是将其增广矩阵通过初等变换变成除最后一列之外只含有0和1的矩阵即可(因为这样子的矩阵所对应的方程,一般情况下,每个方程的左侧都是一个系数为1的未知数,其它未知数的系数都变成0了,那么该未知数自然就等于右侧那个常数;然后每行中1所在的列不同,那么对应的就是不同的未知数了)。

初等变换一般记为$\sim$,它不是等于,初等变换不是等于号的含义,它们是不同的。矩阵变换后,和原矩阵也是同型矩阵,但是两个同型矩阵相等必须每个元素都相同,而矩阵变换的含义是它们对应的线性方程组同解,变换之后的矩阵和原来的矩阵的对应元素不一定相等,因此变换后的矩阵不一定等于原来的矩阵。

\begin{example}
    使用初等变换解线性方程组
    \begin{equation}
        \begin{cases}
            3x+2y=5 \\
            10x-8y=11
        \end{cases}
    \end{equation}
\end{example}
\begin{solution}
    设$A$为该线性方程组的增广矩阵,
    则
    $A= \begin{pmatrix}
        3 & 2 & 5 \\ 
        10 & -8 & 11 
    \end{pmatrix}$,

    $A \sim
    \begin{pmatrix}
        \frac{3}{2} & 1 & \frac{5}{2} \\ 
        -\frac{5}{4} & 1 & -\frac{11}{8} 
    \end{pmatrix}
    \sim
    \begin{pmatrix}
        \frac{11}{4} & 0 & \frac{31}{8} \\ 
        -\frac{5}{4} & 1 & -\frac{11}{8} 
    \end{pmatrix}
    \sim
    \begin{pmatrix}
        1 & 0 & \frac{31}{22} \\ 
        -\frac{5}{4} & 1 & -\frac{11}{8} 
    \end{pmatrix}
    \sim
    \begin{pmatrix}
        1 & 0 & \frac{31}{22} \\ 
        0 & 1 & \frac{17}{44} 
    \end{pmatrix}
    $,
    
    所以,原方程组的解为
    \begin{equation}
        \begin{cases}
            x=\frac{31}{22} \\
            y=\frac{17}{44}
        \end{cases}
    \end{equation}
\end{solution}

上例中,最后一个矩阵称之为\textbf{行最简型矩阵},因此求解线性方程组就是将对应的增广矩阵通过初等变换变成行最简型矩阵。

行最简型矩阵就是,它是\textbf{行阶梯型矩阵}的特例。行阶梯型矩阵就是

\subsection{未知数和方程数量不同的线性方程组}
之前我们所讨论的是未知数和方程数量相同的线性方程组(这里相同指的是每个方程都不同,也就是一个方程不能变成另一个方程,例如“x=2”和“3x=6”算相同的方程),
很明显,它有且仅有一个解($x,y,...$只是这个解的分量,它还是这个解的一部分,因此是\textbf{一个}解,而不是\textbf{一组}解,只有多个解才称为一组解。